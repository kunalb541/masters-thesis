% Abstract for Master's Thesis

\cleardoublepage
\thispagestyle{empty}

\vspace*{50pt}

\section*{Abstract}

The upcoming era of large-scale time-domain surveys---notably the Vera C.~Rubin Observatory's Legacy Survey of Space and Time (\lsst) and the Nancy Grace \romantel\ Space Telescope---will increase gravitational microlensing event detection rates from $\sim$2,000 to over 20,000 events annually. This dramatic increase necessitates automated classification systems capable of distinguishing binary lens events (planetary systems, stellar binaries) from simple Point-Source Point-Lens (PSPL) events. The challenge lies in early-time classification: binary events masquerade as PSPL during their initial phases, yet timely identification is critical for triggering follow-up observations of transient planetary features.

This thesis presents a machine learning framework for automated binary microlensing classification using one-dimensional Convolutional Neural Networks (\cnn s) with TimeDistributed architecture. We generate one million synthetic light curves spanning realistic parameter ranges, deliberately sampling binary configurations that produce distinctive caustic-crossing signatures. The dataset incorporates adaptable observational effects including survey-specific cadence patterns and photometric noise levels matching OGLE, \lsst, and \romantel\ characteristics.

The TimeDistributed \cnn\ processes sequential temporal windows, enabling dynamic classification as observations accumulate rather than requiring complete light curves. We systematically evaluate performance across observation completeness levels, quantifying the confidence-timeliness trade-off essential for operational deployment. Comprehensive benchmarking against traditional $\chi^2$ fitting establishes performance limits and identifies complementary strengths of each approach.

Results demonstrate that deep learning achieves high classification accuracy with partial light curves while maintaining computational efficiency suitable for real-time survey operations. This work provides actionable recommendations for \lsst\ and \romantel\ alert stream integration and delivers an open-source framework ready for deployment in next-generation microlensing surveys.

\vspace{1cm}

\cleardoublepage
