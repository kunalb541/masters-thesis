\chapter*{Abstract}
\addcontentsline{toc}{chapter}{Abstract}

The Nancy Grace Roman Space Telescope will detect tens of thousands of gravitational microlensing events, with many exhibiting binary lens signatures from star-planet or binary star systems. Early identification of these events is critical for coordinating time-sensitive ground-based follow-up observations to constrain system parameters and detect planetary companions. Traditional classification methods based on $\chi^2$ model fitting are computationally expensive and require complete light curves, limiting their utility for real-time survey operations.

This thesis presents a machine learning classifier that provides real-time classification of microlensing events as they unfold. The CNN-GRU architecture processes Roman-like observations (15-minute cadence, 72-day seasons) and outputs continuously updated probabilities for three event classes: flat, PSPL (single lens), and binary lens systems. This dynamic capability enables rapid decision-making for follow-up strategies, allowing observation requests while events are still developing.

Trained on 600,000 synthetic light curves and validated on 300,000 independent events, the classifier achieves 99\% accuracy on events with clear caustic crossings. Performance naturally decreases on the full event population, reflecting the fundamental physical limit where high-impact-parameter binary events become indistinguishable from single-lens profiles. The system provides sub-millisecond inference, making it suitable for processing thousands of events in real-time alert streams.

The classifier addresses a critical operational need for next-generation microlensing surveys by enabling automated early detection of scientifically valuable binary events. The open-source implementation provides a practical tool for developing Roman alert systems and coordinating observations between space-based and ground-based networks.

\cleardoublepage